%% ------------------------------------------------------------------------- %%
\chapter{Conclus�es}
\label{cap:conclusoes}
Em nossos experimentos percebemos que o par�metro limite igual a 1 causou uma perda de pacotes \textit{CPU}.
Isso ocorreu devido ao processamento cont�nuo de pacotes que, quando reduz a taxa de interrup��es pr�ximo a 0, sobrecarrega o sistema com interrup��es de \textit{software} e este passa a descartar pacotes.

Com limite igual a 2, no \textit{VirtualBox} n�o houve diferen�a em rela��o aos limites 60 e 200, no Xen, o sistema usou mais \textit{CPU} em rela��o aos limites 60 e 200, por fim, no \textit{VMware} houve perda de pacotes semelhante ao que ocorreu com limite igual a 1.
Com limites altos (60 e 200) houve um bom desempenho tanto em uso de \textit{CPU} como largura de banda e n�o houve diferen�a no resultado entre valores altos.

Assim, sabe-se que diferementemente dos dispositivos f�sicos que t�m bons resultados com limites baixos, nos dispositivos virtuais, limites altos tiveram um bom desempenho em todos os casos comparado com limites baixos.
No caso particular, com o e1000, \textit{driver} usado nos experimentos, o valor de limite por padr�o � 64, n�o sendo necess�rio alter�-lo se decidirmos escolher o melhor limite.

Vimos que o uso de \textit{CPU} � sens�vel a mudan�as na frequ�ncia de envio de pacotes mesmo quando a largura de banda n�o varia. 
� prov�vel que controlar a frequ�ncia de envio de pacotes, aumente a frequ�ncia de pacotes processada e, consequentemente, reduza o uso de \textit{CPU} como ocorreu no \textit{VirtualBox} e \textit{Xen} com transmiss�o acima de 800 Mbits/s.

Quando monitoramos o uso de \textit{CPU} da m�quina virtual com o \textit{Xen}, \textit{VirtualBox} e \textit{Vmware}, os resultados mostraram que o \textit{Xen} foi o que usou menos \textit{CPU} e teve o melhor desempenho da largura de banda.

%------------------------------------------------------
\section{Sugest�es para Pesquisas Futuras} 
Durante o experimento configuraramos o tamanho do \textit{buffer} do \textit{socket}.
A escolha do valor para esse par�metro poderia ser automatizado para um valor que n�o use muita mem�ria e n�o descarte pacotes.

Tamb�m selecionamos a \textit{CPU} na qual o \texttt{iperf} seria executado pois estavam comprometendo a medi��o. 
Entre usar uma \textit{CPU} e usar duas, em tarefas que exigem apenas uma \textit{CPU}, houve uma carga extra no processamento quando usamos duas \textit{CPU}.
J� com tarefas que exigem duas \textit{CPUs}, experimentos com apenas uma CPU resultou em perda de pacotes.
� interessante analisar quando os processos poder�o usar uma \textit{CPU} ou mais.


