%% ------------------------------------------------------------------------- %%
\chapter{Introdu��o}
\label{cap:introducao}



% \emph{Thesis are random access. Do NOT feel obliged to read a thesis from beginning to end.}



%% ------------------------------------------------------------------------- %%
\section{Considera��es Preliminares}
\label{sec:consideracoes_preliminares}

Considera��es preliminares\footnote{Nota de rodap� (n�o abuse).}\index{genoma!projetos}.
% index permite acrescentar um item no indice remissivo
Texto texto texto texto texto texto texto texto texto texto texto texto texto
texto texto texto texto texto texto texto texto texto texto texto texto texto
texto texto texto texto texto texto texto.
 

%% ------------------------------------------------------------------------- %%
\section{Objetivos}
\label{sec:objetivo}

Texto texto texto texto texto texto texto texto texto texto texto texto texto
texto texto texto texto texto texto texto texto texto texto texto texto texto
texto texto texto texto texto texto.

%% ------------------------------------------------------------------------- %%
\section{Contribui��es}
\label{sec:contribucoes}

As principais contribui��es deste trabalho s�o as seguintes:

\begin{itemize}
  \item Item 1. Texto texto texto texto texto texto texto texto texto texto
  texto texto texto texto texto texto texto texto texto texto.

  \item Item 2. Texto texto texto texto texto texto texto texto texto texto
  texto texto texto texto texto texto texto texto texto texto.

\end{itemize}

%% ------------------------------------------------------------------------- %%
\section{Organiza��o do Trabalho}
\label{sec:organizacao_trabalho}

No Cap�tulo~\ref{cap:conceitos}, apresentamos os conceitos ... Finalmente, no
Cap�tulo~\ref{cap:conclusoes} discutimos algumas conclus�es obtidas neste
trabalho. Analisamos as vantagens e desvantagens do m�todo proposto ... 

As sequ�ncias testadas no trabalho est�o dispon�veis no Ap�ndice \ref{ape:sequencias}.
