\chapter{Proposta}
\label{cap:proposta}

\section{Tema de Pesquisa}
Essa pesquisa tem como tema técnicas de otimização na virtualização de rede dentro de infraestruturas de nuvem.

\section{Problema de Pesquisa}
O problema a ser tratado nessa pesquisa é o baixo desempenho das infraestruturas em nuvem em aplicações que usam muito a rede.

\section{Evidências do problema}
Infraestruturas que normalmente utilizam técnicas de virtualização, como as infraestruturas em nuvem, têm um desempenho diferente da arquitetura de rede padrão. Alguns passos adicionais são necessários para transmitir e receber um pacote de informação de dentro de uma máquina virtual, que implicam em um custo adicional tanto na memória como no processamento \cite{chaudhary2008comparison} \cite{ekanayake2010high} \cite{liu2010evaluating} \cite{Waldspurger:2012:IV:2063176.2063194} \cite{Rixner:2008:NVB:1348583.1348592}. Na revisão é possível ver melhor as propostas de cada autor.


\section{Relevância do problema}
TODO

\section{Proposta}
    Nessa pesquisa, propomos um algoritmo para automatizar a configuração de tamanho de \textit{buffer} e atenuação de transmissão e recepção do \textit{driver} da placa de rede virtual e física, tentando ajustar esses parâmetros dinamicamente de forma a garantir uma melhor qualidade do serviço da infraestrutura de nuvem. 
A proposta será uma solução se o desempenho em largura de banda e latência de aplicações que usam intensamente a rede for equivalente a uma infraestrutura sem virtualização.
    \cite{dong2011optimizing} propuseram uma otimização por atenuação de transmissão e recepção, também chamado de agrupamento de interrupções, nos \textit{drivers} virtuais. Nessa pesquisa eles focaram em otimizar os \textit{drivers} virtuais, deixando de fora o \textit{driver} físico e o tamanho do \textit{buffer}.

\section{Questão de pesquisa}


\section{Objetivo}
O objetivo dessa pesquisa é fornecem um algoritmo de tamanho de \textit{buffer} e atenuação de transmissão e recepção a fim de garantir a qualidade do serviço prestado pelas infraestrutura de nuvem para aplicações que usam intensamente a rede.

\section{Método de pesquisa}

\section{Cronograma}
