\chapter{Proposta}
\label{cap:proposta}

\section{Tema de Pesquisa}
Essa pesquisa tem como tema o uso da \textit{NAPI} em dispositivos de rede virtuais.

\section{Problema de Pesquisa}
A \textit{NAPI} possui um par�metro chamado limite que � capaz de controlar quando o sistema inicia uma varredura nos pacotes. 
O problema que essa pesquisa prentende resolver � a falta de um mecanismo que indique qual o melhor valor para o par�metro limite para dispositivos virtuais considerando a largura de banda e o uso de \textit{CPU}.

\section{Relev�ncia do Problema}
Diferentemente dos computadores f�sicos, as m�quinas virtuais necessitam executar alguns passos adicionais para transmitir ou receber um pacote de informa��o que implicam em um custo adicional no processamento \cite{chaudhary2008comparison} \cite{ekanayake2010high} \cite{liu2010evaluating} \cite{Waldspurger:2012:IV:2063176.2063194} \cite{Rixner:2008:NVB:1348583.1348592}.
Assim, estrat�gias que reduzem o uso da \textit{CPU} causado pelo tr�fego intensivo de pacotes pela rede s�o importantes para reduzir a carga extra gerada pela virtualiza��o.

Atualmente a \textit{NAPI} � implementada por v�rios \textit{drivers} de dispositivos de rede para agregar interrup��es, por�m, h� poucos estudos na �rea, principalmente quando consideramos dispositivos virtuais. 

Em \cite{NAPI}, � citado que n�o existe uma orienta��o para a escolha do valor de limite e muitos \textit{drivers} definem o valor entre 16 e 64, j� em \cite{salah2009implementation}, um experimento pr�tico com limite igual a 2, 6 e 300 mostrou um desempenho de largura de banda melhor com limite igual a 2, por�m essa medi��o foi feita apenas em um dispositivo f�sico.

\section{Proposta de Pesquisa}
� proposto nessa pesquisa um mecanismo para escolher o melhor valor para o limite que melhore a largura de banda e o uso da \textit{CPU} em dispositivos virtuais.

\section{Quest�o de Pesquisa}
O mecanismo que seleciona o melhor valor para o par�metro limite � capaz de obter um desempenho superior em largura de banda e uso de \textit{CPU} em rela��o ao valor padr�o do definido pelo \textit{driver}?
